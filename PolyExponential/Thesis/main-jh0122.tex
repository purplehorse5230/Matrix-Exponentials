\documentclass[12pt]{article}
\usepackage[utf8]{inputenc}
\usepackage{setspace}
\doublespacing
\usepackage{graphicx}
\graphicspath{ {./image/} }

%\usepackage{minted}
\usepackage{amsmath}
\usepackage{tabularx}
\usepackage{amsthm}
\newtheorem{definition}{Definition}

\usepackage{amsmath,amsfonts,amssymb,graphicx,fancyhdr,latexsym,float,amsthm,bm,graphicx,mathtools,txfonts,mdframed}
\usepackage{enumitem,listings,verbatim,color}

\usepackage[T1]{fontenc}    % use 8-bit T1 fonts
\usepackage{hyperref}       % hyperlinks
\usepackage{url}            % simple URL typesetting
\usepackage{booktabs}       % professional-quality tables
\usepackage{amsfonts}       % blackboard math symbols
%\usepackage{nicefrac}       % compact symbols for 1/2, etc.
%\usepackage{microtype}      % microtypography
\usepackage{xcolor}         % colors
\usepackage{epsfig}
%\usepackage{enumerate}
%\usepackage[ruled,vlined, linesnumbered]{algorithm2e}
\newcommand{\R}{\mathbb{R}}
\newcommand{\argmax}{\operatornamewithlimits{argmax}}
\newcommand{\argmin}{\operatornamewithlimits{argmin}}
\newcommand{\bS}{\mathcal{S}}
\newtheorem{theorem}{Theorem}
\usepackage{geometry}
%\usepackage{algorithm,algpseudocode}

% Set custom margin values
\geometry{
    left=1.25in,      % Set left margin to 2cm
    right=1in,     % Set right margin to 2cm
    top=1in,       % Set top margin to 2cm
    bottom=1in     % Set bottom margin to 2cm
}
\DeclareMathOperator{\tr}{tr}

\newcounter{chunk}
\parskip=12pt
\parindent=0mm

\pagestyle{plain}



\title{On the Accuract and Efficient Approximation of Matrix Exponentials}


\author{Emily H. Huang}
\date{2023.9-2024.5}

\begin{document}

\maketitle
A senior honors thesis submitted to the faculty of the University of North Carolina at Chapel Hill in fulfillment of the requirements for the Honors Carolina Senior Thesis in the Department of Mathematics.


\newpage
\section*{Abstract}
Abstract Here.

\newpage
\section*{Acknowledge}
I extend my sincere gratitude to Professor Yifei Lou. As my research mentor in the UNC Department of Mathematics, 
she has provided me with insightful feedback, expert guidance, and invaluable support that have been indispensable 
to the success of my thesis. Her dedication, patience, and encouragement have been a constant source of 
inspiration and have greatly shaped the quality of my work. She has patiently answered my countless questions, 
helping me to understand complex theorems and concepts. Her unwavering support and guidance have not only helped 
me in my academic pursuits but also in my personal life. I feel incredibly fortunate to have had the opportunity 
to work with such an exceptional mentor, and I will always be grateful for her mentorship.\par
I also want to express my gratitude to my colleagues who have supported me during my academic journey. 
Your enthusiasm, advice, and encouragement have been instrumental in keeping me motivated and on track. 
I am grateful for our discussions, where we share ideas, compare our numerical results, share research resources, 
and discuss any problems we encounter.  The resources and facilities provided by the University of North Carolina 
at Chapel Hill have also been crucial to the success of my research. \par
Lastly, I owe a debt of gratitude to my family and friends for their unwavering love, support, and understanding 
during this rigorous journey. Their encouragement has been invaluable, and I am grateful for their contributions 
to this work. Thank you for making this journey both meaningful and rewarding.



\newpage
\tableofcontents
\newpage
\section{Introduction}

\subsection{Examples of Motivation}

\begin{figure}[h]
  \centering
    % \includegraphics[width=8cm]{alcohol.jpg}
    \caption{How to add a picture}
\end{figure}

\subsection{Research Questions}
\begin{figure}[h!]
  \centering
  \begin{minipage}[b]{0.45\linewidth}
    % \includegraphics[width=\textwidth]{dep.jpg}
    \caption{Dependent data}
	\label{fig:research1}
  \end{minipage}
  \hspace{0.5cm}
  \begin{minipage}[b]{0.45\linewidth}
    % \includegraphics[width=\textwidth]{indep.jpg}
    \caption{Independent data}
	\label{fig:research2}
  \end{minipage}
\end{figure}

\subsection{Scientific Challenges}

\subsection{Explorations and Contributions}

\section{Matrix Exponential and Operator Splitting} 
Many physical processes can be decomposed as the combination of simple processes,
e.g., many biological processes are modeled by the reaction-diffusion models, and the 
Navier-Stokes equation contains advection, rection, and diffusion processes. Each
process is well studied both analytically and numerically, the challenging research questions
are how different components interact/compete with each other to produce complex but interesting
phenomena? And how the complex process can be simulated using existing efficient tools built 
for each individual and well-studied process.

In this research, we consider a matrix version of the challenges.

Assume a physical process is modeled by the exponential of matrix $e^{-Ht}$ where $H$
is a size $m \times m$ matrix containing contributions from different physical processes 
and $t$ is the scalar time parameter. Assume $H$ can be decomposed as the sum of 
two simple matrices
$$H=V+T$$ where $V$ is a low-rank matrix and $T$ is a diagonal matrix.
Clearly, both $e^{-Vt}$ and $e^{-Tt}$ are easy to calculate, as 
\begin{theorem}
	The exponential of a diagonal matrix is a diagonal matrix.
\end{theorem}

\begin{theorem}
	The exponential of a low-rank matrix is the Identity matrix plus a low-rank matrix.
\end{theorem}

Unfortunately, because matrix multiplication is not commutative, in general
$$e^{-Ht} \neq e^{-Tt}e^{-Vt}.$$

There are different approaches to utilize the simple structures in $V$ and $T$ to compute the more complex
$e^{-Ht}$. One such approach is to use the product of matrix exponentials to achieve higher order approximation
of $e^{-Ht}$ \cite{}. In this thesis, we explore a polynomial based approach and approximate
$$e^{-Ht} \approx p_n(Ht)$$ 
where $p_n(x)$ is a polynomial of degree $n$. Note that as $Y(t)=e^{-Ht}$ satisfies the
ordinary different equations
\begin{equation}
	\left\{
		\begin{array}{l}
			Y'(t)= - H \cdot Y(t) \\
			Y(0) = I_{m\times m}
		\end{array}
	\right.
\end{equation}
As there exist no polynomial with a bounded degree that exactly satisfy the 
differential equation, we therefore search for a degree $n$ polynomial that
satisfies a pseudo-spectral (collocation) formulation by requiring the 
differential equation is exactly satisfied at $n+1$ collocation points.

{\noindent \bf Pseudo-spectral Formulation} \\
For a given set of collocation points $\{ t_1, t_2, \cdots, t_{n+1}\}$, 
the pseudo-spectral formulation finds a polynomial matrix $p_n(t)$ 
which satisfies
\begin{equation}
	\left\{
		\begin{array}{l}
			p_n'(t_j)=-H \cdot p_n(t_j) \\
			p_n(\cdot 0) = I_{m\times m}
		\end{array}
	\right.
\end{equation}

Unfortunately, the spectral differentiation is an ill-conditioned operator, as demonstrated 
by the following numerical experiments.

{\noindent \bf (Pseudo-)Spectral differentiation is ill condition} \\
Problem setting: Giving the function values $f(x_j)$ at $\{ t_1, t_2, \cdots, t_{n+1}\}$,
one can construct an interpolating polynomial $p_n(x)$. Differentiate the polynomial, one 
can approximation $f'(x) \approx p_n'(x)$. In this experiment, we compare the analytical 
$f'(x)$ with its numerical approximation $p_n'(x)$ at the collocation points 
$\{ t_1, t_2, \cdots, t_{n+1}\}$.

The code is written using Mathematica and is available at .

\section{Integral Equation Reformulation for Matrix Exponential Problem}
\label{sec:iem}
Unlike the spectral differentiation operation discussed in the previous section, the
spectral and pseudo-spectral integration operators are numerically very stable, as deomonstrated by
the following experiment.

{ \noindent \bf (Pseudo-)Spectral integration operator is well-conditioned} \\
Problem setting: Giving the function values $f(x_j)$ at $\{ t_1, t_2, \cdots, t_{n+1}\}$,
one can construct an interpolating polynomial $p_n(x)$. Differentiate the polynomial, one 
can approximation $\int_{t=0}^x f(t)dt \approx \int_0^x p_n'(t) dt$. In this experiment, 
we compare the analytical anti-derivative $\int f(x)$ with its numerical approximation 
$\int p_n(x)$ at the collocation points $\{ t_1, t_2, \cdots, t_{n+1}\}$.

The code is written using Mathematica and is available at .

Therefore instead of the original differential equation formulation, we consider the following Picard
integral equation reformulation
\begin{equation}
	\left\{
		\begin{array}{l}
			Y(x)= Y(0) + \int_0^x (-H) \cdot Y(t) dt \\
			Y(0) = I_{m\times m}
		\end{array}
	\right.
\end{equation}
and searching for a polynomial matrix $p_n(t)$ that satisfies the discretized 
pseudo-spectral formulation
\begin{equation}
	\label{eq:iem}
	\left\{
		\begin{array}{l}
			p_n(t_j)=p_n(0) + \int_0^{t_j} (-H) \cdot p_n(\tau) d\tau \\
			p_n(0) = I_{m\times m}
		\end{array}
	\right.
\end{equation}

\section{Spectral Deferred Correction Methods}
To solve the pseudo-spectral formulation \ref{eq:iem}, we apply the spectral deferred correction (SDC) 
approach to improve the efficiency of the algorithm. 

{\noindent \bf Step 1:} The first step of the SDC is to find an approximation polynomial 
solution $\tilde{Y}(t)$ using a low-order method.
To demonstration the ideas, we simply apply the first order method from matrix exponentials and compute
$\tilde{Y}(t)$ as follows.

$$\tilde{Y}(0) = I_{m\times m}. $$
$$\tilde{Y}(t_{j+1}) = e^{-T (t_{j+1}-t_j)}e^{-V (t_{j+1}-t_j)} \tilde{Y}(t_{j}) $$

{\noindent \bf Comment:} Instead of the first order approximation, one can also apply higher order approximations,
e.g., the 2nd order Strang spliting.
In Rachel's work, many such higher order splittings are developed. An interesting question is how different ``low-order"
predictors will change the efficiency of the algorithm, which will be studied in some details in this thesis.

Once the approximate solution $\tilde{Y}(t)$ is available and define $Y(x)= \tilde{Y}(x) + \delta(x)$,
plug in the Picard integral equation
\begin{equation}
	\left\{
		\begin{array}{l}
			\tilde{Y}(x) + \delta(t)= Y(0) + \int_0^x (-H) \cdot (\tilde{Y}(x) + \delta(t) ) dt \\
			\delta(0) = 0_{m\times m}
		\end{array}
	\right.
\end{equation}
one can derive a new set of equations for the error (also called defect) $\delta(t)$ 
\begin{equation}
	\left\{
		\begin{array}{l}
			\delta(t)=  \int_0^x (-H) \cdot \delta(t) dt + 
			  \left( Y(0) + \int_0^x (-H) \cdot \tilde{Y}(x)  dt -\tilde{Y}(x) \right) \\
			\delta(0) = 0_{m\times m}
		\end{array}
	\right.
\end{equation}
Defining the residue 
$$\epsilon(x)= Y(0) + \int_0^x (-H) \cdot \tilde{Y}(x)  dt -\tilde{Y}(x),$$
the error's equation becomes a new Picard integral equation 
\begin{equation}
	\left\{
		\begin{array}{l}
			\delta(t)=  \int_0^x (-H) \cdot \delta(t) dt + \epsilon(t) \\
			\delta(0) = 0_{m\times m}
		\end{array}
	\right.
\end{equation}
Note that $\tilde{Y}$ is known, the integral $\int_0^x (-H) \cdot \tilde{Y}(x)  dt$
can be accurately evaluated using the very high order (and stable) pseudo-spectral integration matrix.

{\noindent \bf Step 2:} The second step of the SDC method is to apply a low-order method to get a low-order 
estimate $\tilde{\delta}(t)$ of the analytical error (or defect) $\delta(t)$.
Define stepsize $h_j=t_j-t_{j-1}$, $h_0=t_1-0$, and approximate the integral using the Trapezoidal rule, we have
$$\tilde{\delta}(t_1)=(-H) (\frac{h_1}{2} \tilde{\delta}(t_1) )+\epsilon(t_1),$$
$$\tilde{\delta}(t_k)=(-H) \left( \sum_{j=1}^k \frac{h_j}{2} (\tilde{\delta}(t_j)+\tilde{\delta}(t_{j-1})) 
    \right)+\epsilon(t_j).$$
Moving all the terms with $\tilde{\delta}(t_k)$ to the left, at each 
time step, one needs to solve the matrix equation system
\begin{equation}
	\label{eq:loworder}
(I + \frac{h_k}{2} H) \tilde{\delta}(t_k) =RHS(t_k)
\end{equation}
where all the known terms are collected in the term $RHS(t_k)$.

To design a low order method which solves Eq.~(\ref{eq:loworder}) efficiently, we consider the
following approximation
$$(I + \frac{h_k}{2} H)^{-1} \approx I - \frac{h_k}{2} H \approx e^{- \frac{h_k}{2} H}.$$
Therefore, the same low-order time splitting schemes can be applied. 

Note that the previous approach is only first order accurate, to investigate possible higher
order schemes, we consider the differential equation form of the Picard integral equation
\begin{equation}
	\left\{
		\begin{array}{l}
			\delta'(t)= - H \cdot \delta(t) + \epsilon'(t) \\
			\delta(0) = I_{m\times m}
		\end{array}
	\right.
\end{equation}
The analytical solution is given by 
$$\delta(t_{j+1})=e^{-Ht} \delta(t_j) + \int_{t_j}^{t_{j+1}} e^{-H(t- \tau)} \epsilon'(\tau) d \tau.$$
Applying integration by parts, we have 
$$\delta(t_{j+1})=e^{-H h_{j+1}} \delta(t_j) + e^(-H(t-\tau)) \epsilon(\tau) |_{\tau=t_j}^{t_{j+1}} 
  + \int_{t_j}^{t_{j+1}} H e^{-H(t- \tau)} \epsilon(\tau) d \tau$$
where $h_{j+1}=t_{j+1}-t_j$. Therefore
$$\delta(t_{j+1})=e^{-H h_{j+1}} \delta(t_j) + \epsilon(t_{j+1})- e^{-H h_{j+1}} \epsilon(t_j)
  + \int_{t_j}^{t_{j+1}} H e^{-H(t- \tau)} \epsilon(\tau) d \tau.$$
A higher order (but not spectral order) quadrature rule can be applied to 
evaluate $\int_{t_j}^{t_{j+1}} H e^{-H(t- \tau)} \epsilon(\tau) d \tau$.
When the trapezoidal rule is applied, the update formula becomes
$$\delta(t_{j+1})=e^{-H h_{j+1}} \delta(t_j) + \epsilon(t_{j+1})- e^{-H h_{j+1}} \epsilon(t_j)
  + \frac{h_{j+1}}{2} ( H e^{-H h_{j+1}} \epsilon(t_j) +  H \epsilon(t_{j+1})).$$
In the numerical implementation, an exponential expansion based low order method can be applied to
approxiate $e^{-H h_{j+1}}$ and be evaluated efficiently. 

For the given approximate solution $\tilde{Y}$, the final outcome (output) is the low-order approximation
of the error (defect) $\tilde{\delta}(t)$. This can be explictly represented as
$$\tilde{\delta}(t) = ImpFun(\tilde{Y}).$$

{\noindent \bf Step 3:} There are two different approaches to continuous improve the approximate solution.
In the first approach, a better estimate of the approximate solution is simply 
$$\tilde{Y}_{new} = \tilde{Y}_{old} + \tilde{\delta}$$ 
and one can repeat step 2 until the iterations are convergent or a maximum number of iterations is reached.
In numerical linear algebra language, this fixed-point (stationary) iterations represent a particular 
Neumann series expansion. The resulting algorithm is referred to as the spectral deferred correction (SDC) method 
in existing literature.

In the second approach, instead of a naive Neumann series expansion, one can use the terms in the Neumann series to
construct a Krylov subspace and search for the optimal solution in the Krylov subspace. The resulting algorithms
are well-studied by the numerical linear algebra community. Instead of a detailed review of the mathematical 
foundation and existing implementations, we refer interested readers to \cite{}. In our implementation, 
as most matrices are in general non-symmetric, existing GMRES, restarted GMRES, Transpose-free QMR, and
BiCGStab have been used to improve the convergence properties of the SDC approach caused by a few bad 
eigenvalues. The resulting algorithm is referred to as the Krylov deferred correction method (KDC) \cite{}.
The implementation of KDC is a simple application of existing Krylov subspace methods to the linear equation
system $\tilde{\delta}(t) = ImpFun(\tilde{Y})$ where the matrix vector multiplication result is given by
$\tilde{\delta}(t)$ and one is search for the root of $ImpFun(Y)=0$, i.e., when the pseudo-spectral solution
becomes the input of the implicit function, the output low order solution should be $\tilde{\delta}(t) =0$.

\subsection{SDC vs. KDC}
The advantage of the SDC method is that one only needs the results from previous step (not all historical
data) in order to start a new round of refinement, therefore requiring minimal storage. This is normally
the right choice when there are no convergence issues (due to a few bad eigenvalues in the Neumann series
expansion). 

When there are bad eigenvalues, Neumann series may converge slowly (order reduction) or become divergent.
In this case, by searching for the optimal solution in the Krylov subspace using least squares,
the KDC method will converge more efficiently once the bad eigenvalues are fully resolved. However, this
approach requires additional operations and the storage of historical data. 

Finding the ``optimal" method for a particular problem is always a challenging research topic and the answer 
depends on a lot of factors, including both the problem properties and computer hardware resources.
Some sample factors one needs to consider in the problem discussed in this thesis include the stepsize,
the eigenvalue distributions of $H$, $V$, and $T$, the accuracy and efficiency of the low-order scheme,
the special structures (low-rank, symmetry) in the matrices, and a lot more.

\section{Numerical Experiments}

\section{Future work}


\newpage 
\bibliographystyle{plain}
\setcounter{page}{1}
%\renewcommand{\thepage}{E-\arabic{page}}
\bibliography{Ref}
%\bibliography{Ref,KZ_BIB_031223,huang}

\end{document}
